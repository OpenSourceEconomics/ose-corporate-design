%%%%%%%%%%%%%%%%%%%%%%%%%%%%
%%  FONT SETTINGS THESIS  %%
%%%%%%%%%%%%%%%%%%%%%%%%%%%%


\usepackage{amsfonts, amssymb}  % Additional symbols from AMS

\usepackage{extarrows}

% Use Times Roman for text and math
\usepackage{newtxtext}
\usepackage[smallerops, varg, upint, slantedGreek]{newtxmath}

%\usepackage[scale=1.07]{tgcursor}
%	% Use Courier as the typewriter font
%\usepackage[scale=0.93]{sourcecodepro}
%	% Use Source Code Pro as the typewriter font
\usepackage[scaled=0.835]{DejaVuSansMono}  % Scaled to match the x-height of the newtxtext Times Roman
	% Use DejaVu Sans Mono as the typewriter font, also supports Greek via the LGR encoding

\usepackage{xfrac}

\usepackage[protrusion=true, expansion=false, kerning=true]{microtype}
% This package enables so-called hanging punctuation. That is, when a punctuation sign like ":", ".", "-", etc. is found at the beginning or end of a line, it is protruded a little into the page margin. This results in "optical margin alignment," because the protrusion makes the margin alignment look straighter.
%\DisableLigatures[f]{family = {sf*, rm*}}
%	% Disable the f* ligatures for both Fira Sans and Charter because both fonts provide insufficient support
%\DisableLigatures[f]{family = {rm*}}
	% Disable the f* ligatures for Charter because it provides insufficient support
\SetExtraKerning[unit=space]
{encoding=*, family=*, series=*, size={*, normalsize, footnotesize}, font = */*/*/*/*}
{\textemdash = {325, 325},
	/ = {100, 100},
	: = { 50,   0},
	; = { 50,   0}}
\renewcommand{\textellipsis}{\mbox{.{\kern.09em}.{\kern.09em}.}}

\renewcommand{\bibfont}{%
	\normalsize%
	\addtolength{\baselineskip}{0.05\baselineskip}%
	%\setlength{\baselineskip}{\newbaselineskip}%
}  % Use same size for bibliography as for body text.

\captionsetup{footfont={rm, footnotesize}}  % Set font in \floatfoot to serif and footnotesize




%%%%%%%%%%%%%%%%%%%%%%%%%%%%%%%%
%%  MATH-RELATED ADJUSTMENTS  %%
%%%%%%%%%%%%%%%%%%%%%%%%%%%%%%%%


\usepackage{mathrsfs}  % Provides the \mathscr math alphabet

% The following is taken from
% https://tex.stackexchange.com/questions/116389/automatic-upright-math-when-text-is-in-italic/116399#116399
% Filling in ``missing'' Greek glyphs for completeness
% (not really necessary, since they look identical to Latin glyphs and are thus almost never used)
% ==>
\newcommand{\omicron}{o}
\newcommand{\Digamma}{F}
\newcommand{\Alpha}  {A}
\newcommand{\Beta}   {B}
\newcommand{\Epsilon}{E}
\newcommand{\Zeta}   {Z}
\newcommand{\Eta}    {H}
\newcommand{\Iota}   {I}
\newcommand{\Kappa}  {K}
\newcommand{\Mu}     {M}
\newcommand{\Nu}     {N}
\newcommand{\Omicron}{O}
\newcommand{\Rho}    {P}
\newcommand{\Tau}    {T}
\newcommand{\Chi}    {X}
% <==

% Upright Greek letters for the \mathup command
% ==>
\makeatletter
% Save original definitions of the Greek letters
\@for\@tempa:=%
	alpha,beta,gamma,delta,epsilon,zeta,eta,theta,iota,kappa,lambda,mu,nu,xi,%
	omicron,pi,rho,sigma,varsigma,tau,upsilon,phi,chi,psi,omega,digamma,%
	Alpha,Beta,Gamma,Delta,Epsilon,Zeta,Eta,Theta,Iota,Kappa,Lambda,Mu,Nu,Xi,%
	Omicron,Pi,Rho,Sigma,Tau,Upsilon,Phi,Chi,Psi,Omega,Digamma%
	\do{%
		\expandafter\let\csname\@tempa orig\expandafter\endcsname\csname\@tempa\endcsname%
		\expandafter\let\csname\@tempa uporig\expandafter\endcsname\csname\@tempa up\endcsname%
	}%
\newcommand*{\upgreekletters}{%
	\@for\@tempa:=%
		alpha,beta,gamma,delta,epsilon,zeta,eta,theta,iota,kappa,lambda,mu,nu,xi,%
		omicron,pi,rho,sigma,varsigma,tau,upsilon,phi,chi,psi,omega,digamma,%
		Alpha,Beta,Gamma,Delta,Epsilon,Zeta,Eta,Theta,Iota,Kappa,Lambda,Mu,Nu,Xi,%
		Omicron,Pi,Rho,Sigma,Tau,Upsilon,Phi,Chi,Psi,Omega,Digamma%
		\do{%
			\expandafter\let\csname\@tempa\expandafter\endcsname\csname\@tempa up\endcsname%
		}%
}
\newcommand*{\itgreekletters}{%
	\@for\@tempa:=%
		alpha,beta,gamma,delta,epsilon,zeta,eta,theta,iota,kappa,lambda,mu,nu,xi,%
		omicron,pi,rho,sigma,varsigma,tau,upsilon,phi,chi,psi,omega,digamma,%
		Alpha,Beta,Gamma,Delta,Epsilon,Zeta,Eta,Theta,Iota,Kappa,Lambda,Mu,Nu,Xi,%
		Omicron,Pi,Rho,Sigma,Tau,Upsilon,Phi,Chi,Psi,Omega,Digamma%
		\do{%
			\expandafter\let\csname\@tempa\expandafter\endcsname\csname\@tempa orig\endcsname%
		}%
}
\makeatother
% <==

\newcommand{\mathup}[1]{\upgreekletters\mathrm{#1}\itgreekletters}

\renewcommand{\mathbf}[1]{\bm{#1}}
\newcommand{\mathbfit}[1]{\mathbf{\mathit{#1}}}
\newcommand{\mathbfup}[1]{\upgreekletters\mathbf{\mathrm{#1}}\itgreekletters}

% Since math mode uses a different font encoding, issuing \euro/\texteuro in math mode
% produces an incorrect sign. We fix this. ==>
\AtBeginDocument{%
	\newcommand*{\euro}[1]{%
		\relax\ifmmode\text{\texteuro}#1\else\texteuro #1\fi%
	}%
}
% <==