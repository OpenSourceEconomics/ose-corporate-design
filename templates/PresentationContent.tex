%%%%%%%%%%%%%%%%%%%%%%%%%%%%%%%%%%%%%%%
%%  MISCELLANEOUS SETTINGS/COMMANDS  %%
%%%%%%%%%%%%%%%%%%%%%%%%%%%%%%%%%%%%%%%




% Add separator slide to the beginning of each section ==>
%% Blue on white:
%\AtBeginSection[]{%
%	\begin{frame}[standout, c]{~}
%		%\vfill
%		\usebeamerfont{title}%
%		\textcolor{SpotColor}{\insertsectionhead}
%		%\vfill
%	\end{frame}%
%}
% White on blue:
\AtBeginSection[]{{%
	\setbeamercolor{background canvas}{bg=OSEBlue, fg=white}%
	\colorlet{SpotColor}{white}%
	\begin{frame}[standout, c]{~}
		%\vfill
		\usebeamerfont{title}\hypersetup{linkcolor=white}%
		\insertsection
		%\vfill
	\end{frame}%
	\colorlet{SpotColor}{OSEBlue}%
}}
%% Alternativelyy, add ``Outline'' slide to the beginning of each section ==>
%\AtBeginSection[]{
%	\begin{frame}[plain]{Outline}
%		\tableofcontents[currentsection]
%	\end{frame}
%}
%% <==

\title{Academic Presentations}

\subtitle{A~LaTeX Template Using the \texttt{beamer} Class}

%\subtitle{I~Prefer to Avoid Subtitles}

\author[Smith, Smith, and Smith]{%
	Adam Smith\inst{a,\,c} \and
	% Set the name of the presenting coauthor in boldface:
	\alert{Janet Smith}\inst{b,\,c} \and
	Jeremiah Smith\inst{a}
} % Author(s)

\institute{%
	\inst{a}\,University of Open Source, Germany \and
	\inst{b}\,University of Economics Bonn, Germany \and
	\inst{c}\,Institute for Applied Microeconomics%
} % Institution(s)

\date{%
	Inviting Institution/Seminar Series \\[\medskipamount]
	\textmd{\today}%
}




%%%%%%%%%%%%%%%%%%%
%%  TITLE SLIDE  %%
%%%%%%%%%%%%%%%%%%%


\begin{frame}[standout]{~}

	\titlepage%

\end{frame}


\begin{frame}[standout]{Outline}

	\medskip
	\tableofcontents

\end{frame}




%%%%%%%%%%%%%%%%%
%%  MAIN PART  %%
%%%%%%%%%%%%%%%%%

\section{Introduction}


\begin{frame}{\titleprefix: Choice of a~Reasonable Aspect Ratio}

	When preparing a~presentation, we often do not know whether the native aspect ratio of the projector in the seminar room/lecture hall will be 4\,:\,3 or 16\,:\,9 (or 16\,:\,10).
	
	In this case, it may be a~good idea to choose an~\alert{intermediate aspect ratio}, see \url{https://github.com/josephwright/beamer/issues/497}. The idea behind this recommendation is that it minimizes the average loss of available space.
	
	Hence, these templates include a~presentation in the \alert{14\,:\,9 aspect ratio} (see \url{https://en.wikipedia.org/wiki/14:9_aspect_ratio}): while it is imperfect for probably every projector that you will encounter, it is good on average for all of them.
	
	(Please note that 14\,:\,9${}\mathrel{\dot{=}} 1.556$, which is pretty close to the ``officially'' recommended 20\,:\,13${}\mathrel{\dot{=}} 1.5385$.)

\end{frame}


\begin{frame}{\titleprefix}

\begin{quote}
	Great Minds Discuss Ideas. \\
	Average Minds Discuss Events. \\
	Small Minds Discuss People. \\
	\upshape ---\url{https://quoteinvestigator.com/2014/11/18/great-minds/}
\end{quote}

\heading{Background}
\begin{itemize}
	\item Temporal discounting is key concept in economics.
	\item Normative model: exponential discounting. However, observed decisions are hard to explain 
	\item Robust decision-making is the greatest thing on earth.
\end{itemize}

\end{frame}


\begin{frame}{\titleprefix}

\heading{Research Question}
\begin{itemize}
	\item The composition of latex and of typical rubbers is given below.
	\item Is it true that trees are regularly tapped and the coagulated latex which exudes is collected and worked up into rubber?
\end{itemize}

\pause

\heading{Preview of the Results}
\begin{itemize}
	\item There is no feasible method at present known of preventing the inclusion of the resin of the latex with the rubber during coagulation.
	\item[$\Rightarrow$\hspace{-2.5pt}] Although the separation of the resin from the solid caoutchouc by means of solvents is possible, it is not practicable or profitable commercially.
\end{itemize}

\end{frame}


\begin{frame}{\titleprefix: Beamer \texttt{block} Environments}

\begin{block}{Block title example: 0123456789 äöüß ÄÖÜ Often finding flowers in official fjords}
	The \texttt{block} environment. The \texttt{block} environment. The \texttt{block} environment. The \texttt{block} environment. The \texttt{block} environment. \insertblocktitle.\strut  % \strut so that the bottom rule is not too close to the text
\end{block}%

\begin{exampleblock}{An~exemplary example}
	I~am the \texttt{exampleblock} environment. Use me for examples.\strut  % \strut so that the bottom rule is not too close to the text
\end{exampleblock}

\begin{alertblock}{Summary: Things to remember}
	The \texttt{alertblock} environment. Use this environment for really important stuff. The \texttt{alertblock} environment.\strut  % \strut so that the bottom rule is not too close to the text
\end{alertblock}

\end{frame}


\begin{frame}{\titleprefix: \texttt{block} Environment with Different Colors}

\begin{block}{A~\texttt{block} in the default color}
	The \texttt{block} environment. The \texttt{block} environment. The \texttt{block} environment. The \texttt{block} environment. \insertblocktitle.\strut  % \strut so that the bottom rule is not too close to the text
\end{block}%

{%
	\renewcommand{\framedblockcolor}{UBonnYellow}%
	\begin{block}{A~\texttt{block} in yellow}
		The \texttt{block} environment. The \texttt{block} environment. The \texttt{block} environment. The \texttt{block} environment. \insertblocktitle.\strut  % \strut so that the bottom rule is not too close to the text
	\end{block}%
}

\begin{block}{A~\texttt{block} in the default color}
	The \texttt{block} environment. The \texttt{block} environment. The \texttt{block} environment. The \texttt{block} environment. \insertblocktitle.\strut  % \strut so that the bottom rule is not too close to the text
\end{block}%

\end{frame}


\begin{frame}{\titleprefix: \texttt{definition} and \texttt{theorem} Environments}

\begin{definition}[A~Very, Very, Very, Very, Very, Very Long Name of a~Concept that Spans Two Lines]
	The \texttt{definition} environment. Upright.
\end{definition}

\begin{theorem}[Theorem's mame]
	The \texttt{theorem} environment. Italic.
\end{theorem}%

\begin{lemma}[Lemma's Name]
	The \texttt{lemma} environment. Italic.
\end{lemma}%

\begin{corollary}[Corollary's Name]
	The \texttt{corollary} environment. Italic.
\end{corollary}%	

\begin{proof}[Proof of Theorem's Name]
	The \texttt{proof} environment. Upright.
\end{proof}

\end{frame}


\begin{frame}{\titleprefix: Design of the Study}

\begin{itemize}
	\item The latex of the best rubber plants furnishes from 20\% to 50\% of rubber.
	\item As the removal of the impurities of the latex is one of the essential points to be aimed at, it was thought that the use of a centrifugal machine to separate the caoutchouc as a cream from the watery part of the latex would prove to be a satisfactory process.
\end{itemize}

\end{frame}



\begin{frame}{\titleprefix: Design of the Study}

The watery portion of the latex soaks into the trunk, and the soft spongy rubber which remains is kneaded and pressed into lumps or balls:

\begin{tabularx}{\textwidth}{@{} l @{\hspace{0.67em}} L @{}}
	\alert{Robust} &
	Each payment transferred on single day. \\
	\addlinespace
	\alert{Decision} &
	Earlier payoff concentrated, while later payoff dispersed over ${n = 2}$, $4$, or $8$~dates. \\
	\addlinespace
	\alert{Making} &
	Earlier payoff dispersed over ${n = 2}$, $4$, or $8$ dates, while later payoff concentrated.
\end{tabularx}

\end{frame}


\begin{frame}{\titleprefix: Control Experiment}

\begin{itemize}
	\item Control for alternative explanations.
	\item Many of the example sentences were taken from \url{http://sentence.yourdictionary.com/latex}.
\end{itemize}

\end{frame}


\begin{frame}{\titleprefix: An~Example \texttt{enumerate} List}

\blindlistlist[3]{enumerate}[4]

\end{frame}


\begin{frame}{\titleprefix: An~Example \texttt{itemize} List}

\blindlistlist[3]{itemize}[4]

\end{frame}


\begin{frame}{\titleprefix: Some Example Text}

	\heading{Let's include some Greek letters: $\alpha$, $\beta$, $\sigma$}
	
	\blindtext $\alpha$, $\beta$, $\sigma$
	
	\textsf{Test: \fontencoding{LGR}\selectfont qrst qrs t qrs\noboundary\ t}. Math mode, upright: $\mathord{\textup{\fontencoding{LGR}\selectfont s\noboundary}}$

\end{frame}


\begin{frame}{\titleprefix: Some Example Formulas}

	\alert{Let's include some additional Greek letters: $\gamma, \phi, \sigma_\epsilon, c^\alpha$}
	\vspace{-\smallskipamount}
	\[
		p(R, \phi) \sim
			\int_{-\infty}^\infty
				\frac
					{ \tilde{W}_n(\gamma) \exp \left[ \imath R / a \left( \sqrt{k^2 a^2 - \gamma^2} \cos \phi \right) \right] }
					{ (k^2 a^2 - \gamma^2)^{3/4} {H'}_n^{(1)} \left( \sqrt{k^2 a^2 - \gamma^2} \right) }
				\mathup{d}\gamma
	\]
	\pause
	\vspace{-\smallskipamount}
	\alert{Let's also include some upright Latin letters in math mode: $\mathup{d}$, $\mathup{e}$ (\hyperlink{Eulers_number}{next slide})}
	\[
		\int_{a}^{b} f(x)\,\mathup{d}x = F(b) - F(a)
	\]
	\pause
	\vspace{-\medskipamount}
	\alert{Let's test the math bold style}
	\[
		\mathbfup{\Sigma} \coloneqq
		\mathup{Cov}(\mathbf{X}) =
		\begin{bmatrix}
			\mathup{Var}(X_1)      & \cdots & \mathup{Cov}(X_1, X_n) \\[-2.5pt]
			\vdots                 & \ddots & \vdots                 \\
			\mathup{Cov}(X_n, X_1) & \cdots & \mathup{Var}(X_n)
		\end{bmatrix}
	\]

\end{frame}


\begin{frame}{\titleprefix: Additional Example Formulas (with upright $\mathup{\pi}$)}

	\def\Pr{\ensuremath{\mathbb{P}}}
	\def\rmd{\mathup{d}}
	Only variables are set in italics according to \caps{ISO} style---hence, we use upright ``$\rmd$,'' ``$\mathup{e}$,'' and ``$\mathup{\pi}$'' (\texttt{\textbackslash mathup\{d\}}, \texttt{\textbackslash mathup\{e\}}, and \texttt{\textbackslash mathup\{\textbackslash pi\}}, respectively).
	
	\begin{theorem}[simplest form of the \emph{Central Limit Theorem}]
		\ifnum \serifbodyfont=0
			\sffamily
		\fi
		Let $X_1, X_2, \cdots$ be a sequence of i.i.d. random variables with mean $0$ 
		and variance $1$ on a~probability space $(\Omega, \mathcal{F}, \Pr)$. Then
		\hypertarget{Eulers_number}{}
		\[
			\Pr\left(\frac{X_1+\cdots+X_n}{\sqrt{n}}\le y\right) \to
			\mathfrak{N}(y) \coloneqq 
			\int_{-\infty}^y \frac{\mathup{e}^{-v^2/2}}{\sqrt{2\mathup{\pi}}}\,\mathup{d}v
			\quad\text{as} \quad n\to\infty,
		\]
		or, equivalently, letting $S_n \coloneqq \sum_1^n X_k$,
		\[
			\mathbb{E} f\left(S_n/\sqrt{n}\right) \to
			\int_{-\infty}^\infty f(v) \frac{\mathup{e}^{-v^2/2}}{\sqrt{2\piup}}\,\mathup{d}v
			\quad \mbox{as $n\to\infty$, for every $f \in \mathup{b}\mathcal{C}(\mathbb{R})$.}
		\]
	\end{theorem}

\end{frame}


\begin{frame}{\titleprefix: Overview}

\begin{enumerate}
	\item<1-> As a secondary function we may recognize the power of closing wounds, which results from the rapid coagulation of exuded latex in contact with the air:
	\begin{enumerate}
		\item<2-> In some cases (Allium, Convolvulaceae, etc.) rows of cells with latex-like contents occur.
		\item<2-> However, the walls separating the individual cells do not break down.
	\end{enumerate}
	\item<3-> The rows of cells from which the laticiferous vessels are formed can be distinguished (6.3~p.p. vs. 2.6~p.p.; ${p < 0.01}$).
\end{enumerate}

\end{frame}


\begin{frame}{\titleprefix: An Automated Animation}

The automated transition to the next slide (=~page in the PDF document) only works in full-screen mode.
\begin{itemize}
	\item The feature is available in Adobe Acrobat and Acrobat Reader.
	\item Unfortunately, it is (currently, \today) not available in macOS Preview, Skim, and SumatraPDF.
\end{itemize}
\transduration{0.25}%
\only<12>{\transduration{}}\hypertarget<1>{animation_start}{}%
\foreach \n [evaluate=\n as \angle using \n * 30] in {0, ..., 12}{
	\only<\n>{
		\begin{figure}
			\begin{tikzpicture}
			\draw[draw=none, use as bounding box](-1, 0) rectangle (1, 2);
			\filldraw[fill=SpotColor, draw=none] (0,1) -- (0,2) arc (90:90-\angle:1cm) -- cycle;
			\end{tikzpicture}
			\caption{Step~\n---Angle: \angle\textdegree}
		\end{figure}
	}
}%
\vspace{-\bigskipamount}
\hyperlink<12>{animation_start}{\beamerreturnbutton{Back to the start}}

\end{frame}


\begin{frame}[allowframebreaks]{\titleprefix: Testing the \texttt{allowframebreaks} option}

Let's test automatic numbering with the \texttt{allowframebreaks} option.

On this slide, \textbf{no} number should be included in the frame title.

Random cite: \cite{Knuth1984}

\end{frame}


\begin{frame}[allowframebreaks]{\titleprefix: Testing the \texttt{allowframebreaks} Option}

\renewcommand{\blindmarkup}[1]{\emph{#1}}

Let's test automatic numbering with the \texttt{allowframebreaks} option.

On this slide, ``(1/3)'' should appear in the frame title.

\blindtext

\parstart{\framebreak}
\Blindtext[2]

\end{frame}


%%%%%%%%%%%%%%%%%%
%%  REFERENCES  %%
%%%%%%%%%%%%%%%%%%


\section{\refname}
%\renewcommand\bibsection{}
%	% To prevent ``References'' from appearing twice in the navigation bar

\begin{frame}[allowframebreaks]{\insertsection}

\begin{refcontext}[sorting=nyt]
	
	% Sort BIBLIOGRAPHY by alphabet (while CITATIONS are sorted by year)
	\printbibliography[heading=none]
\end{refcontext}

%% Only when using BibTeX instead of BibLaTeX ==>
%\bibliographystyle{0_0_Preamble/aea_doi_url_href}
%\bibliography{Library}
%% <==

\end{frame}



%%%%%%%%%%%%%%%%
%%  APPENDIX  %%
%%%%%%%%%%%%%%%%

\begin{appendix}
	
	
	\section[Appendix\newline \textmd{Backup Slides}]{Appendix}
	
	
	\begin{frame}[label=model]
	
	\frametitle{\insertsection: Modeling Concentration Bias}
	
	Subjects consider a sequences of consequences $\boldsymbol{c}$ from choice set $\boldsymbol{C}$.
	
	%Focusing theory augments discounted utility (constant, hyperbolic, \dots) through an~additional weight \highlight{$g[\cdot]$} on the instantaneous utility function.
	\begin{itemize}
		
		\item \alert{Standard discounted utility:}
		Suppose that the instantaneous utility function $u$ satisfies ${u'>0}$ and ${u''\leq 0}$, and that earlier consequences are preferred over later consequences of the same magnitude, i.e., ${D(t)\leq 1}$:
		\item[] ${U}(\boldsymbol{c}\phantom{, \boldsymbol{C}}) \coloneqq
		\sum_{t=1}^{T} \phantom{g_t} D(t)\,u(c_t)$, \quad where, e.g., \quad $D(t) = \delta^t$  or $D(t) = \frac{1}{1 + k\,t}$. %$\beta \delta^t$.
		\medskip
		\item \only<1->{\alert{Focusing model \citep{Koszegi2013}:}}
		\item[]<1-> $\tilde{U}(\boldsymbol{c}\highlight{, \boldsymbol{C}}) \coloneqq \sum_{t=1}^{T} \highlight{g_t}\,D(t)\,u(c_t)$, \quad where \\[3pt]
		\highlight{$g_t \equiv % g[\Delta_t(C)]$, $\Delta_t(C) =
			g[\max_{\boldsymbol{c}'\in \boldsymbol{C}} %D(t) 
			u(c'_t) - \min_{\boldsymbol{c}'\in \boldsymbol{C}} %D(t) 
			u(c'_t)]$}
		\smallskip
		\begin{itemize}
			\item<1-> Weighting function \highlight{$g[\cdot]$} increases in difference of maximum and minimum possible utility at a~point in time.
			\item<1-> Subjects overweight intertemporal consequences with a greater range.
			%\item<3-> Subjects overweight intertemporal consequences with a greater range 
		\end{itemize}
	\end{itemize}
	
\end{frame}


\end{appendix}
